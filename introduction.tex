% \clearpage
% \section*{INTRODUCTION}
% \addcontentsline{toc}{chapter}{Introduction}
\chapter{Introduction}

Computer simulations allow us to explore ``what if'' situations by building a representative model that contains many of the parameters of the actual physical system, but in a virtual form. In this way, costly, time-consuming, or unsafe physical experiments may be avoided and, instead cheap and efficient in-silica ``experiments'' may be performed. Simulations can be used to develop a deeper understanding of a system and can support experiments by generating hypotheses or finding optimal values for input variables.

The goal of many simulations is to evaluate the effect of changing variables in a system, often to determine optimal values. A key advantage to simulation over experiment is the ability to capitalize on existing mathematical and computational techniques to determine optimal values without the need to test every possible value.

Another benefit to simulations is their ability to simulate non-physical processes. For example, in Chapter 2 below we enhance the speed of a simulation by giving atoms non-physical properties. According to the laws of statistical mechanics the final results are physically relevant, even though the intermediate steps are not.

The topic of the first part of this dissertation is free energy calculation in molecular systems. The free energy is a fundamental physical quantity that determines whether a system will spontaneously change between states, and more specifically, gives the relative amount of time a system will spend in each state. Estimating the free energy for molecular system is a powerful tool, e.g., to determine how amino acid mutations modify protein folding or binding, or for developing new drugs. A long-standing goal of our research group is to develop and test methods for efficient free energy computation. In Chapter 2 we have implemented the previously-developed adaptive integration method (AIM) in a popular molecular modeling software package. We tested our implementation on two molecular systems and show that our results converge to a higher level of accuracy and precision for a given simulation time as compared to standard methods.

The molecular modeling used in Chapter 2 moves all the atoms in the system by calculating interactions at the atomic level. While this is a valuable approach for studying small molecular systems it is not useful for studying large-scale systems such as mechanical stress on an engine. Thus, in Chapter 3 we used the finite element method where the surface of any physical object is broken into a mesh of smaller parts called finite elements. Forces between objects are then approximated by summing the forces for all elements. Smaller elements provide greater accuracy and allow us to represent more complex geometries.

The purpose of the research in Chapter 3 is to examine the force configurations of alternating arrangements of permanent magnet arrays, called Halbach arrays. The ultimate goals is to design a superconducting flywheel energy storage system for use in deep space. For this study we used two-dimensional finite element analysis in order to determine the optimal configuration for our permanent magnet arrays. These simulations allowed us to find the optimum configuration of magnets by exploring every possible magnet configuration in a matter of hours instead of spending months in a lab. 
