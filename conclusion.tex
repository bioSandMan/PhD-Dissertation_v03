\clearpage
\section*{CONCLUSION}
\addcontentsline{toc}{chapter}{Conclusion}

Computational simulations allow us to explore "what if" situations by building a mathematical model which contains all of the parameters of the physical system in virtual form. In this way, costly, time-consuming or unsafe experiments may be avoided and performed "in-silica" cheaply and in a time efficient manner. The power of computational simulation is in the ability to facilitate the understanding of a system's behavior without actually testing the system in the real world. In addition, simulation can support experimentation where simulation represents systems or generates data needed to meet experiment objectives. 

Within the first part of this dissertation we showed a particular method, AIM, to be superior to the other methods in the context of what was tested. Further more, we found ways to improve the design of our simulation by showing that the density of intermediate states is directly related to the agreement between the methods, e.g. there must be sufficient sampling of the lambda states for convergence. In fixed lambda simulations the problem lies in the density of the chosen lambda states. Some states will contribute disproportionately to the variance of the estimate, therefore testing short simulations of different lambda densities before attempting longer simulations is a more economical use of time and resources. We found that running a longer simulation in a state space that is not dense enough to fully describe the state function propagates sampling error in regions of high variance. In contrast, lambda density should be increased in regions of high variance. By not increasing the density of states one is not able to achieve the required  ‘smoothness’ of the function in regions of high variation. This was a very important find for our future experiments where we will explore larger protein systems and more complex mutations.

For the second part of this dissertation we used 2D Finite Element Analysis in order to determine the optimal configuration for our permanent magnet arrays. These simulations allowed us to massively reduce the amount of work needed to perform our experiments. We were able to find the optimum configuration of magnets by programmatically exploring every possible configuration in a matter of hours instead of spending months in a lab. We note that the simulation is a simplification, i.e. a 2D Finite Element Analysis is limited in its ability to predict the interactions between magnetic flux penetration and High Temperature Superconductors. However, exploring every possible combination of permanent magnet arrays in the lab is nearly impossible. The simulations allowed us to reduce the costs of creating physical equipment to hold the magnets in place and expensive equipment to accurately measure the magnetic fields produced by the arrays. We stipulate that the simulation data is only appropriate for initial guess work and tells us very little of the underlying physics. In order to more accurately account for flux penetration and leakage, a 3-dimensional, non-linear solver is required. This would be something appropriate for future work in this study.